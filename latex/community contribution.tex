
\documentclass{article}
\usepackage[landscape]{geometry}
\usepackage{url}
\usepackage{multicol}
\usepackage{amsmath}
\usepackage{esint}
\usepackage{bigints}
\usepackage{amsfonts}
\usepackage{xcolor}
\usepackage{tikz}
\usetikzlibrary{calc}
\usetikzlibrary{decorations.pathmorphing}
\usepackage{amsmath,amssymb}

\usepackage{colortbl}
\usepackage{xcolor}
\usepackage{mathtools}
\usepackage{amsmath,amssymb}
\usepackage{enumitem}
\usepackage{xhfill}
\makeatletter

\newcommand*\bigcdot{\mathpalette\bigcdot@{.5}}
\newcommand*\bigcdot@[2]{\mathbin{\vcenter{\hbox{\scalebox{#2}{$\m@th#1\bullet$}}}}}
\makeatother

\title{física eléctrica}
\usepackage[brazilian]{babel}
\usepackage[utf8]{inputenc}
%cambios de unidades (pt, mm, cm,ex,em,bp,dd,pc,sp) https://tex.stackexchange.com/questions/8260/what-are-the-various-units-ex-em-in-pt-bp-dd-pc-expressed-in-mm
\advance\topmargin-.8in
\advance\textheight3in
\advance\textwidth3in
\advance\oddsidemargin-1.5in
\advance\evensidemargin-1.5in
\parindent0pt
\parskip2pt
\newcommand{\hr}{\centerline{\rule{3.5in}{1pt}}}
%\colorbox[HTML]{e4e4e4}{\makebox[\textwidth-2\fboxsep][l]{texto}
\newcommand{\nc}[2][]{%
	\tikz \draw [draw=black, ultra thick, #1]
	($(current page.center)-(0.5\linewidth,0)$) -- 
	($(current page.center)+(0.5\linewidth,0)$)
	node [midway, fill=white] {#2};
}% tomado de https://tex.stackexchange.com/questions/179425/a-new-command-of-the-form-tex
\begin{document}
	
	\begin{center}{\huge{\textbf{Statistical Tests and Parameter Estimations in R}}}
	
	Candong Chen cc4766\\
	\end{center}
	\begin{multicols*}{3}
		
		\tikzstyle{mybox} = [draw=black, fill=white, very thick,
		rectangle, rounded corners, inner sep=10pt, inner ysep=10pt]
		\tikzstyle{fancytitle} =[fill=black, text=white, font=\bfseries]
		
		%--------------------------
		\begin{tikzpicture}
			\node [mybox] (box){%
				\begin{minipage}{0.3\textwidth}
					%\addtolength{\itemsep}{-2pt}
					\begin{itemize}
						\item List the estimations $$E[X^k]=g_k(\theta_1,\cdots,\theta_m),k=1,\cdots,m$$
						\item Solve the equation set $$\theta_k=\theta_k(E[X],E[X^2],\cdots,E[X^m])$$
						\item Replace $E[X^k]$ with $M_k=\frac{1}{n}\sum_{i=1}^{n}X_{i}^{n}$, we get the moment estimations $$\hat{\theta}_{k}=\theta_{k}(M_1,\cdots,M_m)$$
					\end{itemize}
					
					{\tt rootSolve::multiroot(f, start, maxiter = 100)}\\
					
					Given n (nonli­near) equations, we use multiroot to solve for n roots.
					
					1. {\bf f} is the euqation set w.r.t. estima­tors.
					
					2. {\bf start} is the initial guesses for unknown variables of {\bf f}.
					
					3. {\bf maxiter} is maximal number of iterations allowed.
				\end{minipage}
			};
			%---------------------------------
			\node[fancytitle, right=10pt] at (box.north west) {Moment Estimation};
		\end{tikzpicture}
		
		%---------------------------
		\begin{tikzpicture}
			\node [mybox] (box){%
				\begin{minipage}{0.3\textwidth}
					\begin{itemize}
						\item 
						Build likelihood function $$L=P(X_1=x_1,\cdots,X_n=x_n)=\Pi_{i=1}^{n}f(x_{i};\theta)$$
						
						\item MLE of $\theta$ is $\hat{\theta}=\arg\max_{\theta} \log L$
						
						\item Solve the equation $\frac{d\log L}{d\theta}=0$ and verify if $\frac{d^2 \log L}{d\theta^2}<0$
					\end{itemize}
					{\tt op­tim­­(par, fn, method = c("N­­el­d­e­r-­­Mea­­d", "­­BF­G­S­", "­­CG­", "­­L-­B­F­GS­­-B", "­­SA­N­N­", "­­Br­e­n­t"), lower = -Inf, upper = Inf)}\\
					
					1. {\bf optim} calculates maximum likelihood estimates for multiple parameters distri­but­ions.
					
					2. {\bf par} sets the initial value of parame­ter.
					
					3. {\bf fn} is the likelihood function.
					
					4. {\bf method} provides six ways to calculate extremum.
					
					5. {\bf lower} and {\bf upper} are the lower and upper bounds of parameters respec­tively
				\end{minipage}
			};
			%---------------------------------
			\node[fancytitle, right=10pt] at (box.north west) {MLE};
		\end{tikzpicture}
		
		%---------------------------
		
		
		%---------------------------
		\begin{tikzpicture}
			\node [mybox] (box){%
				\begin{minipage}{0.3\textwidth}
					$L(\theta;X)=\int p(Z|X,\theta)p(X|\theta)$
					
					Iteratively applying these two steps:
					\begin{itemize}
						
						\item Expectation step (E step): $$Q(\theta|\theta^{(t)})=E_{Z|X,\theta^{(t)}}[\log L(\theta;X,Z)]$$
						\item Maximization step (M step): $$\theta^{(t+1)}=\arg\max_{\theta}Q(\theta|\theta^{(t)})$$
						
					\end{itemize}
					{\tt mclust­­::­M­c­lu­­st(­­data, G = NULL, modelNames = NULL, prior = NULL, control = emCont­­rol(), initia­­li­z­ation = NULL)}\\
					
					1. {\bf G} specifies the number of catego­rie­s.The default is G=1:9.
					
					2. {\bf modelNames} specifies the fitting model during the EM algorithm.
					
					3. {\bf prior} allows specif­ication of a conjugate prior on the means and variances through the function priorC­ontrol.
					
					4. {\bf control} specifies the control parameters for the EM algorithm.
				\end{minipage}
			};
			%------------ campo titulo---------------------
			\node[fancytitle, right=10pt] at (box.north west) {EM Algorithm};
		\end{tikzpicture}
		%---------------------------------
		%\bigskip
		%---------------------------
		\begin{tikzpicture}
			\node [mybox] (box){%
				\begin{minipage}{0.3\textwidth}
					\begin{tabular}{ll}
						Empirical Distribution & {\tt ecdf(x)}\\
						Histogram & {\tt hi­st(x, breaks)}\\
						KDE & {\tt density(x, bw = "nrd0",}\\
						 & {\tt adjust, kernel})\\
					\end{tabular}
				
					In KDE:
					
					1. {\bf bw} is the smoothing bandwidth to be used.
					
					2. the bandwidth used is actually {\bf adjust*bw}.
					
					3. {\bf kernel} specifies the smoothing kernel to be used.
					
				\end{minipage}
			};
			%------------ potencial titulo  ---------------------
			\node[fancytitle, right=10pt] at (box.north west) {Distribution Estimation};
		\end{tikzpicture}
		
		%---------------------------
		\begin{tikzpicture}
			\node [mybox] (box){%
				\begin{minipage}{0.3\textwidth}
					\nc{Normal Distribution}
					t test (mean):
					
					{\tt t.test(x, y=NULL, altern­ati­ve=­c("t­wo.s­id­ed",­"­les­s","g­rea­ter­"), mu=0, paired­=FALSE, var.eq­ual­=FALSE, conf.l­eve­l=0.95)	}\\
					
					F test (variance):
					
					{\tt var.te­st(x, y, ratio = 1, altern­ative = c("t­wo.s­id­ed", "­les­s", "­gre­ate­r"), conf.level = 0.95)}
					
					\nc{Binomial Distribution}
					binomial test(p):
					
					{\tt binom.t­est(x, n, p = 0.5, altern­ative = c("t­wo.s­id­ed", "­les­s", "­gre­ate­r"), conf.level = 0.95)}
					
					
				\end{minipage}
			};
			%---------------------------------
			\node[fancytitle, right=10pt] at (box.north west) {Parametric Test};
		\end{tikzpicture}
		
		%\bigskip\\
		%---------------------------
		\begin{tikzpicture}
			\node [mybox] (box){%
				\begin{minipage}{0.3\textwidth}
					{\tt cor.test(x, y, alternative = c("two.sided", "less", "greater"), method = c("pearson", "kendall", "spearman"), exact = NULL, conf.level = 0.95, ...)}\\
					
					1. Pearson's correlation coefficient has a precondition that x and y are
					normally distributed.
					
					2. Spearman's correlation coefficient is used for continuous variables.
					
					3. Kendall's correlation coefficient is used for ordinal categorical variables.
				\end{minipage}
			};
			%---------------------------------
			\node[fancytitle, right=10pt] at (box.north west) {Bivariate Correlation Test};
		\end{tikzpicture}
		
		%\bigskip\\
		%---------------------------

		%---------------------------------

		
		%\bigskip\\
		%---------------------------
		\begin{tikzpicture}
			\node [mybox] (box){%
				\begin{minipage}{0.3\textwidth}
					\nc{Pearson's chi-sq­uared test}
					{\tt chisq.t­est(x, y = NULL, correct = TRUE, p = rep(1/­len­gth(x), length­(x)), rescale.p = FALSE}\\
					
					which is used to test if x and y have the same distri­bution.
					
					1. {\bf correct} is a logical variable that indicates whether it is used for continuous correction.
					
					2. {\bf p} is the theoretical probability that the original hypothesis
					falls in the intervals, and the default value represents uniform distri‐
					bution.

					
					3. {\bf rescale.p} is a logical variable. FALSE (default) requires that sum of the input {\bf p} equals 1.  When TRUE is selected, this condition is not required and the program recalculates the {\bf p} value.
					
					\nc{Shapirowilk test}
					{\tt shapiro.test(x)}\\
					
					which is used to test if x is normally distributed.
					
				\end{minipage}
			};
			%---------------------------------
			\node[fancytitle, right=10pt] at (box.north west) {Nonparametric Test (completely known)};
		\end{tikzpicture}
		
		%\bigskip\\
		%---------------------------
		\begin{tikzpicture}
			\node [mybox] (box){%
				\begin{minipage}{0.3\textwidth}
					\nc{Kolmogorov-Smirnov test}
					{\tt ks.test(x, y, ..., alternative = c("two.sided", "less", "greater"), exact = NULL)}\\
					
					which is used to test if x and y have the same distri­bution.
					
					1. {\bf y} can be either data or a character string specifying distribution such as {\bf ''pnorm''} or {\bf ''pexp''}.
					
					2. {\bf ...} are parameters of the distribution specified (as a character string) by y.
					
					3. {\bf exact} is a logical indicating whether an exact p-value should be computed.
					
					\nc{Contingency table test}
					To test if the two columns in x are independent\\
					
					{\tt chisq.test(x, correct = False)}\\

					the minimum theoretical frequency: $T$ 
					
					the total frequency: $N$\\
					
					Pearson's chi-squared test can be used only if $T\geq 5$ and $N\geq 40$.\\
					
					{\tt chisq.test(x,correct = True)}\\
					
					Pearson's chi-squared modified test should be used if $1\leq T<5$ and $N\geq 40$.\\
					
					{\tt fisher.test(x, alternative =''two.sided''}\\
					
					Fisher exact test should be used if $T<1$ or $N<40$.
					

					
				\end{minipage}
			};
			%---------------------------------
			\node[fancytitle, right=10pt] at (box.north west) {Nonparametric Test (unknown parameters)};
		\end{tikzpicture}
	
			%\bigskip\\
			%---------------------------
		\begin{tikzpicture}
			\node [mybox] (box){%
				\begin{minipage}{0.3\textwidth}
					\nc{Sign test}
					{\tt binom.test(sum(x>M), length(x), alternative = c("less", "greater"), conf.level=0.95)}\\
					
					which is used to test if the median
					of X is greater (less) than M.\\
					
					In the sign test method, only the number of symbols is counted, without considering the magnitude of the absolute value of each sign difference.
					
					\nc{Wilcoxon signed-rank test (univariate)}
					{\tt wilcox.test(x, mu=M, alternative = c("two.sided", "less", "greater"), exact=FALSE, correct=FALSE)
}\\
					
					which is used to test if the median of x is equal (greater or less) to M.
					
					\nc{Wilcoxon signed-rank test (multivariate)}
					{\tt wilcox.test(x, y, alternative = c("two.sided", "less", "greater"), paired = TRUE, correct = TRUE, conf.level = 0.95)
}\\
					
					which is used to test if the median of x is equal (greater or less) to the median of y.
					
					
				\end{minipage}
			};
			%---------------------------------
			\node[fancytitle, right=10pt] at (box.north west) 	{Nonparametric Test (completely known)};
		\end{tikzpicture}	
	\end{multicols*}
\end{document}


